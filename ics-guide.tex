\documentclass{article}

\def\cmd#1{$\backslash$\texttt{#1}}

\begin{document}
\title{Author's Guide to the Symposium on Innovations in Computer Science
  Template \texttt{(ics.cls)}}

\author{Ruini Xue\\
  xueruini@gmail.com}

\maketitle

\begin{abstract}
  This guide explains how to use the ICS \LaTeX{} template. Since the template
  follows \LaTeX{} conventions, it's quite easy to use. Besides, we borrow some
  texts from \texttt{sigplan-guide.pdf}.
\end{abstract}

\section{Introduction}
\label{sec:introduction}
The ICS style is a \LaTeX\ class file that you use to prepare papers for
\emph{Symposium on Innovations in Computer Science} conference proceedings. 

The ICS class file is a variant of the standard \LaTeX\ article style.  It is
based on \texttt{article.cls} and both replaces and adds to its features. This
author’s guide assumes you are familiar with \LaTeX\ and describes the features
of the ICS class file that are new or different.

\subsection{What You Need}
\label{sec:what-you-need}

You only need three files to use the ICS class file: 

\begin{itemize}
\item The \LaTeX{} class file, \texttt{ics.cls}
\item This document, \texttt{\expandafter{\jobname}.pdf}
\item A sample file, \texttt{sample.tex}, to help you get started preparing your
  paper.
\end{itemize}

\section{Document Prolog}
\label{sec:document-prolog}
This chapter describes the commands used in the prolog of your paper. The prolog
is the portion of the \LaTeX{} source file that precedes the text of the paper.

\subsection{Example}
\label{sec:example}

\begin{verbatim}
\documentclass{ics} 

% \usepackage{mypackage}

\begin{document} 
... text of the paper ... 
\end{document}
\end{verbatim}

\subsection{The Document Class}
\label{sec:document-class}
The \cmd{documentclass} command names the ICS class file and lists any desired
options. By now, we do not provide any specific option.

\subsection{Packages}
\label{sec:packages}
If you need to use any \LaTeX\ packages, these are specified immediately following
the \cmd{documentclass} command. Table~\ref{tab:packages} lists the packages
that are used by the ICS class.

\begin{table}
  \centering
  \caption{Packages used by the ICS class file.}
  \label{tab:packages}
  \begin{tabular}{ll}
    \hline
  Name & Description \\
   \hline
  amsmath,amssymb & AMS Math package \\
  graphicx & insert graphics \\
  paralist & refined lists \\
  subfig, caption & (sub)float captions \\
  times & text fonts \\
  natbib & bibliography and reference format \\
  \hline
  \end{tabular}
\end{table}

\subsection{Definitions}
\label{sec:definitions}
If you need any macro definitions for your paper, these should appear immediately
before the \verb|\begin{document}| command that indicates the start of the text
  of the paper. It is best to use \cmd{newcommand} to defined macros, rather
  than \cmd{def}, to ensure that existing macros are not accidently redefined.

\subsection{Texts}
\label{sec:texts}

The text of your paper is enclosed between \verb|\begin{document}| and
  \verb|\end{document}| environment that follows the
prolog. Section~\ref{sec:title-assoc-inform} describes the commands that produce
the title, and author lists on the first page of the paper. These commands
appear first in the text.

\section{Title and Associated Information}
\label{sec:title-assoc-inform}
This chapter describes the commands used to produce the title page of your 
paper.

\section{Example}
\label{sec:example-title}
The following example presents many of the commands necessary to produce 
the title page of your paper. Subsequent sections will discuss the commands in 
detail.

\begin{verbatim}
\begin{document}
\title{A Sample of Symposium on Innovations in Computer Science}

\author{%
  Ruini Xue$^{1}$ 
  \and 
  Wenguang Chen$^{1}$ 
  \and 
  Hong Jiao$^{2}$}

\address{%
  $^{1}$Tsinghua University,Beijing 100084, China
  \and
  $^{2}$Tsinghua Press, Beijing 100084, China}

\email{%
  xueruini@gmail.com
  \and
  cwg@tsinghua.edu.cn
  \and
  jiaoh@tup.tsinghua.edu.cn}

\begin{abstract} 
  ...abstract texts...
\end{abstract}

\keywords{word1; word2; word3; and word4}

%Important: do call \maketitle after abstract and keywords!!!
\maketitle
\end{verbatim}

% \subsection{Conference Information}
% \label{sec:conf-inform}

\subsection{Title}
\label{sec:title}
The following commands are used to provide the title for 
your paper.

\begin{itemize}
\item \verb|\title{title texts}|
\end{itemize}

This command specifies the title of your paper. You can use the linebreak
(\verb|\\|) command to break lines in the title. 

\subsection{Authors}
\label{sec:authors}

The following command is used to specify authors and related information of your paper.

\begin{itemize}
\item \verb|\author{author1 \and author2...}|
\end{itemize}

This command specifies the names of authors of your paper. Use \cmd{and} to
separate different authors\footnote{\cmd{and} is also used to separate
affiliations and emails.}. If the authors' affiliations are different, group
authors of the same institute with superscript as following:

\begin{verbatim}
\author{%
  Ruini Xue$^{1}$ 
  \and 
  Wenguang Chen$^{1}$ 
  \and 
  Hong Jiao$^{2}$}
\end{verbatim}

\begin{itemize}
\item \verb|\address{address1 \and address2...}|
\end{itemize}

This command specifies the affiliations of the authors. If they are grouped in
the \cmd{author} command, please make sure the superscripts are consistent.

\begin{itemize}
\item \verb|\email{email1 \and email2...}|
\end{itemize}

This command specifies the emails of the authors. Different from the previous
two commands, you'd better list each author's email, so do not use superscripts
to group emails here.

\begin{itemize}
\item \verb|\begin{abstract}...\end{abstract}|
\end{itemize}

This environment includes the abstract for your paper.

\begin{itemize}
\item \verb|\keywords{word1; word2; word3...}|
\end{itemize}

This command specifies the key words of your paper. Separate individual key
words with semicolon. This command 

\begin{itemize}
\item \cmd{maketitle}
\end{itemize}

This command must appear following the keywords in order to typeset the top of 
the title page.

\section{Hierarchy}
\label{sec:hierarchy}
Table~\ref{tab:headings} describes the commands that are used to produce hierarchical 
headings in your paper. They strictly follows the \LaTeX\ conventions, so their
corresponding star forms produce non-numbered headings.

\begin{table}
  \centering
  \caption{Headings Hierarchy.}
  \label{tab:headings}
  \begin{tabular}{ll}
    \hline
  Command & Level \\
   \hline
  \verb|\section{title}| & 1 \\
  \verb|\section{title}| & 2 \\
  \verb|\section{title}| & 3 \\
  \verb|\paragraph{title}| & 4 \\
  \verb|\subparagraph{title}| & 5 \\
  \hline  
  \end{tabular}
\end{table}

\section{Figures and Tables}
\label{sec:figures-tables}

A numbered floating figure is coded in this fashion: 

\begin{verbatim}
\begin{figure} 
\centering
... content ... 
\caption{Foundational framework of the snork mechanism.} 
\label{fig-ffsm} 
\end{figure} 
\end{verbatim}

This produces a figure with its content at the top and its caption at the bottom.

A numbered floating table is coded as follows: 

\begin{verbatim}
\begin{table} 
\centering
\caption{Critical parameters of the snork mechanism.} 
\label{tab-cpsm}  
\begin{tabular}{...} 
... tabular content ... 
\end{tabular} 
\end{table}
\end{verbatim}
 
This produces a table with its tabular material at the bottom and its caption at 
the top.

\section{Bibliography}
\label{sec:bibliography}
You can write the bibliography one item by one item like this:

\begin{verbatim}
\begin{thebibliography}{99}
\bibitem{xue08} Ruini X, Wenguang C, Weimin Z. CprFS: A User-level File System
  to Support Consistent File States for Checkpoint and Restart. 22nd ACM
  International Conference on Supercomputing (ICS'08), 114-223, June 7-12, 2008.

\bibitem{tex} Knuth D~E. The {\TeX} Book. 15th ed. Reading, MA: Addison-Wesley
  Publishing Company, 1989

\bibitem{xue09} Ruini X, Xuezheng L, Ming W, Zhenyu G, Wenguang C, Weimin Z,
  Zheng Z, Geoffery M~V. MPIWiz: Subgroup Reproducible Replay of MPI
  Applications. 14th ACM SIGPLAN Symposium on Principles and Practice of
  Parallel Programming (PPoPP'09), 251-260, February 14-18, 2009.

\bibitem{Chafik94} {Chafik El Idrissi} M, Roney A, Frigon C, et~al. Measurements
  of total kinetic-energy released to the {$N=2$} dissociation limit of {H}$_2$
  --- evidence of the dissociation of very high vibrational {R}ydberg states of
  {H}$_2$ by doubly-excited states. Chemical Physics Letters, 1994,
  224(10):260--266

\bibitem{MellingerR96} Mellinger A, Vidal C~R, Jungen C.  Laser reduced
  fluorescence study of the carbon-monoxide nd triplet {R}ydberg
  series-experimental results and multichannel quantum-defect analysis.
  J. Chem. Phys., 1996, 104(5):8913--8921

\bibitem{NPB2} Woo A, Bailey D, Yarrow M, et~al. The {NAS} Parallel Benchmarks
  2.0. Technical report, The Pennsylvania State University CiteSeer Archives,
  December~05, 1995. \url{http://www.nasa.org/}
\end{thebibliography}
\end{verbatim}

Our recommendation is to manage and generate the bibliography with Bib\TeX{} as
following:

\begin{verbatim}
\bibliographystyle{unsrtnat}
\bibliography{bibfile}
\end{verbatim}

The printer for ICS proceedings wants the bibliography to be included in 
your main \LaTeX\ file. If you generate the bibliography with Bib\TeX, you should 
then merge the resulting \texttt{.bbl} file into the main file.

\end{document}
