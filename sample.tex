\documentclass{ics}

% import packages here
% \usepackage{foo,bar}

\begin{document}
\title{A Sample of Symposium on Innovations in Computer Science}
\author{Ruini Xue$^{1}$ \and Wenguang Chen$^{2}$}
\address{1. Tsinghua University Beijing 100084, 2. Insititute of High Performance Computing Beijing 100084}
\email{xueruini@gmail.com, cwg@tsinghua.edu.cn}

\begin{abstract} 
  This template follows \LaTeXe{} conventions, and your \TeX{} and \LaTeX{}
  experiences should work here. The only exception is: call
  \texttt{$\backslash$maketitle} after abstract and keywords! And please
  separate key words with semicolon. Finally, please first update \LaTeX{}
  packages if you encounter strange problems.
\end{abstract}

\keywords{word1; word2; word3; and word4}

%% Important: do call \maketitle after abstract and keywords!!!
\maketitle


\section{Introduction}
\label{sec:introduction}
The Institute for Computer Science (ITCS) is a new institute headed by Professor
Andrew Chi-Chih Yao at Tsinghua University. Its mission is to nurture and
develop the talents of students and researchers and to create knowledge for the
advancement of computer science.

\subsection{References Example}
\label{sec:ref-example}
Xue and Chen explored the feasibility of user-level file system in HPC
environment~\cite{xue08}. Donald~\onlinecite{tex} wrote the
masterpiece. Extensive research on system reliability was conducted in the past
several decades~\cite{xue09,Chafik94,MellingerR96,NPB2}.

\subsection{Details}
\label{sec:details}
It also aims to become one of the leading research centers on theoretical
computer science in the world. The Institute already has an elite graduate
student body and enjoys frequent visits by world-renowned scientists all year
round. With a vibrant research environment and impressive research outputs, ITCS
provides an ideal setting for fruitful interactions and collaborations among its
members.

\subsection{Equation Example}
\label{sec:equation-example}
Please check Eq.~\eqref{eq:1}. DO read \textsf{amsmath} document before you
input complicated equations.

\begin{equation}
  \label{eq:1}
  E=mc^{2}
\end{equation}

\subsection{Table Example}
\label{sec:more-more-details-1}
Please check Table~\ref{tab:example}.

\begin{table}[h]
  \centering
  \caption{Table Example}
  \label{tab:example}
  \begin{tabular}{l|l|l}
    \hline
    first & second & third \\
    \hline
    first & second & third \\
    \hline
    first & second & third \\
    \hline
  \end{tabular}
\end{table}

\subsubsection{More Details}
\label{sec:more-details}
The Institute for Computer Science (ITCS) is a new institute headed by Professor
Andrew Chi-Chih Yao at Tsinghua University\footnote{This is a footnote
  example. Do not use footnote if possible.}. Its mission is to nurture and
develop the talents of students and researchers and to create knowledge for the
advancement of computer science.

\paragraph{More More Details}
\label{sec:more-more-details}
It also aims to become one of the leading research centers on theoretical
computer science in the world. The Institute already has an elite graduate
student body and enjoys frequent visits by world-renowned scientists all year
round. With a vibrant research environment and impressive research outputs, ITCS
provides an ideal setting for fruitful interactions and collaborations among its
members.

\subparagraph{The Details}
\label{sec:details-1}
The Institute for Computer Science (ITCS) is a new institute headed by Professor
Andrew Chi-Chih Yao at Tsinghua University. Its mission is to nurture and
develop the talents of students and researchers and to create knowledge for the
advancement of computer science. 

\section{Design}
\label{sec:design}
It also aims to become one of the leading research centers on theoretical
computer science in the world. The Institute already has an elite graduate
student body and enjoys frequent visits by world-renowned scientists all year
round. With a vibrant research environment and impressive research outputs, ITCS
provides an ideal setting for fruitful interactions and collaborations among its
members.

\subsection{Architecture}
\label{sec:architecture}
The Institute for Computer Science (ITCS) is a new institute headed by Professor
Andrew Chi-Chih Yao at Tsinghua University. Its mission is to nurture and
develop the talents of students and researchers and to create knowledge for the
advancement of computer science.

\subsubsection{Implementation}
\label{sec:implementation}
It also aims to become one of the leading research centers on theoretical
computer science in the world. The Institute already has an elite graduate
student body and enjoys frequent visits by world-renowned scientists all year
round. With a vibrant research environment and impressive research outputs, ITCS
provides an ideal setting for fruitful interactions and collaborations among its
members.

\subsection{Figure Example}
\label{sec:ref-figure}
Please check Figure~\ref{fig:example}. The figure may not appear the place where
your code locates, this is because figures and tables are float objects in
\LaTeX, which means that they are can be moved around to generate better
typesetting results.

\begin{figure}
  \centering
  \includegraphics[width=.8\linewidth]{thu-whole-logo}
  \caption{Figure Example}
  \label{fig:example}
\end{figure}

\section*{Acknowledegment}
\label{sec:acknowledegment}
We thank the anonymous reviewers for their useful feedback. The research was
partially supported by Chinese National 973 Basic Research Program under Grant
2007CB310900 and by Tsinghua National Laboratory for Information Science and
Technology. 

% References
% 
% \bibliographystyle{unsrt}
% \bibliography{bibfile}

\begin{thebibliography}{99}
%% Examples:
%\bibitem{journal} Author1, Author2. Paper Title[J]. Journal Name, Year, Volume(Number):pages.
%\bibitem{onlinedata} Author. Report Title[EB/OL]. [publish date]. the link
%\bibitem{website} Web Site Name. the link
%\bibitem{book} Author1, Author2. Book Title[M]. editions. Publish address:publisher,year.

\bibitem{xue08} Ruini X, Wenguang C, Weimin Z. CprFS: A User-level File System
  to Support Consistent File States for Checkpoint and Restart. 22nd ACM
  International Conference on Supercomputing (ICS'08), 114-223, June 7-12, 2008.

\bibitem{tex} Knuth D~E. The {\TeX} Book. 15th ed. Reading, MA: Addison-Wesley
  Publishing Company, 1989

\bibitem{xue09} Ruini X, Xuezheng L, Ming W, Zhenyu G, Wenguang C, Weimin Z,
  Zheng Z, Geoffery M~V. MPIWiz: Subgroup Reproducible Replay of MPI
  Applications. 14th ACM SIGPLAN Symposium on Principles and Practice of
  Parallel Programming (PPoPP'09), 251-260, February 14-18, 2009.

\bibitem{Chafik94} {Chafik El Idrissi} M, Roney A, Frigon C, et~al. Measurements
  of total kinetic-energy released to the {$N=2$} dissociation limit of {H}$_2$
  --- evidence of the dissociation of very high vibrational {R}ydberg states of
  {H}$_2$ by doubly-excited states. Chemical Physics Letters, 1994,
  224(10):260--266

\bibitem{MellingerR96} Mellinger A, Vidal C~R, Jungen C.  Laser reduced
  fluorescence study of the carbon-monoxide nd triplet {R}ydberg
  series-experimental results and multichannel quantum-defect analysis.
  J. Chem. Phys., 1996, 104(5):8913--8921

\bibitem{NPB2} Woo A, Bailey D, Yarrow M, et~al. The {NAS} Parallel Benchmarks
  2.0. Technical report, The Pennsylvania State University CiteSeer Archives,
  December~05, 1995. \url{http://www.nasa.org/}
\end{thebibliography}
\end{document}
