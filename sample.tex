\documentclass{ics}

% import packages here
% \usepackage{foo,bar}

\begin{document}
\title{A Sample of Symposium on Innovations in Computer Science}
\author{%
  Ruini Xue$^{1}$ 
  \and 
  Wenguang Chen$^{1}$ 
  \and 
  Hong Jiao$^{2}$}
\address{%
  $^{1}$Tsinghua University,Beijing 100084, China
  \and
  $^{2}$Tsinghua Press, Beijing 100084, China}
\email{%
  xueruini@gmail.com
  \and
  cwg@tsinghua.edu.cn
  \and
  jiaoh@tup.tsinghua.edu.cn}

\begin{abstract} 
  This template follows \LaTeXe{} conventions, and your \TeX{} and \LaTeX{}
  experience should work here. The only exception is: call
  \texttt{$\backslash$maketitle} after abstract and keywords! And please
  separate key words with semicolon. Finally, please first update \LaTeX{}
  packages if you encounter strange problems. The source of this file for
  \LaTeX{} users may be used as a template. Both \LaTeX{} experts and beginners
  can start from this file.
\end{abstract}

\keywords{word1; word2; word3; and word4}

%% Important: do call \maketitle after abstract and keywords!!!
\maketitle


\section{First Level Heading}
\label{sec:1st-level-heading}
Sample texts Sample texts Sample texts Sample texts Sample texts Sample texts
Sample texts Sample texts Sample texts Sample texts Sample texts Sample texts
Sample texts Sample texts Sample texts Sample texts Sample texts Sample texts
Sample texts Sample texts.

\subsection{Second Level Heading}
\label{sec:2nd-level-heading}
Sample texts Sample texts Sample texts Sample texts Sample texts Sample texts
Sample texts Sample texts Sample texts Sample texts Sample texts Sample texts
Sample texts Sample texts Sample texts Sample texts Sample texts Sample texts
Sample texts Sample texts.

\subsection{Equation Example}
\label{sec:equation-example}
Please check Eq.~\eqref{eq:1}. DO read \textsf{amsmath} document before you
input complicated equations.

\begin{equation}
  \label{eq:1}
  E=mc^{2}
\end{equation}

\subsection{References Example}
\label{sec:ref-example}
Xue and Chen explored the feasibility of user-level file system in HPC
environment~\cite{xue08}. Donald~\cite{tex} wrote the masterpiece. Extensive
research on system reliability was conducted in the past several
decades~\cite{xue09,Chafik94,MellingerR96,NPB2}.

\subsection{Table Example}
\label{sec:table-example}
Please check Table~\ref{tab:example}. Note that this is just a table example,
the template does nothing to prevent you from designing tables much more
complicated than this example, or even simpler than this one.

\begin{table}[h]
  \centering
  \caption{Table Example}
  \label{tab:example}
  \begin{tabular}{l|l}
    \hline
    Name & Age \\
    \hline
    John & 23 \\
    \hline
    Alice & 22 \\
    \hline
  \end{tabular}
\end{table}

\subsubsection{Third Level Heading}
\label{sec:3rd-level-heading}
Sample texts Sample texts Sample texts Sample texts Sample texts Sample texts
Sample texts Sample texts Sample texts Sample texts\footnote{This is a footnote
  example.} Sample texts Sample texts Sample texts Sample texts Sample texts
Sample texts Sample texts Sample texts Sample texts Sample texts.

\paragraph{4th Level Heading}
\label{sec:4th-level-heading}
Sample texts Sample texts Sample texts Sample texts Sample texts Sample texts
Sample texts Sample texts Sample texts Sample texts Sample texts Sample texts
Sample texts Sample texts Sample texts Sample texts Sample texts Sample texts
Sample texts Sample texts.

\subparagraph{5th Level Heading}
\label{sec:5th-level-heading}
Sample texts Sample texts Sample texts Sample texts Sample texts Sample texts
Sample texts Sample texts Sample texts Sample texts Sample texts Sample texts
Sample texts Sample texts Sample texts Sample texts Sample texts Sample texts
Sample texts Sample texts.

\section{Design}
\label{sec:design}
Sample texts Sample texts Sample texts Sample texts Sample texts Sample texts
Sample texts Sample texts Sample texts Sample texts Sample texts Sample texts
Sample texts Sample texts Sample texts Sample texts Sample texts Sample texts

\subsection{Architecture}
\label{sec:architecture}
Sample texts Sample texts Sample texts Sample texts Sample texts Sample texts
Sample texts Sample texts Sample texts Sample texts Sample texts Sample texts
Sample texts Sample texts Sample texts Sample texts Sample texts Sample texts

\subsubsection{Implementation}
\label{sec:implementation}
Sample texts Sample texts Sample texts Sample texts Sample texts Sample texts
Sample texts Sample texts Sample texts Sample texts Sample texts Sample texts
Sample texts Sample texts Sample texts Sample texts Sample texts Sample texts

\subsection{Figure Example}
\label{sec:figure-example}
Please check Fig.~\ref{fig:example}. The figure may not appear the place where
your code locates, this is because figures and tables are float objects in
\LaTeX, which means that they are can be moved around to generate better
typesetting results. If you want to force its position, try \texttt{h} option as
in Table~\ref{tab:example}. Note that if you want to make the figure to span the
columns, use \texttt{figure*} instead of \texttt{figure}. This also works for
tables and user defined floats via package \textsf{float}.

\begin{figure}
  \centering
  \includegraphics[width=.8\linewidth]{thu-whole-logo}
  \caption{Figure Example}
  \label{fig:example}
\end{figure}

\section{Conclusion}
\label{sec:conclusion}
This template is created with the following principle in mind: \textbf{do not
  break the users' \TeX/\LaTeX{} experience}. Therefore, it is expected that
your paper should be compiled successfully with slight modification (recall the
\texttt{$\backslash$maketitle} hint in the abstract).

\section*{Acknowledegment}
\label{sec:acknowledegment}
We thank the anonymous reviewers for their useful feedback. The research was
partially supported by Chinese National 973 Basic Research Program under Grant
2007CB310900 and by Tsinghua National Laboratory for Information Science and
Technology. 


\appendix
\section*{Appendix A}
put appendix contents here.


% References
% 
% \bibliographystyle{unsrt} % or unsrtnat
% \bibliography{bibfile}

\begin{thebibliography}{99}
%% Examples:
%\bibitem{journal} Author1, Author2. Paper Title[J]. Journal Name, Year, Volume(Number):pages.
%\bibitem{onlinedata} Author. Report Title[EB/OL]. [publish date]. the link
%\bibitem{website} Web Site Name. the link
%\bibitem{book} Author1, Author2. Book Title[M]. editions. Publish address:publisher,year.

\bibitem{xue08} Ruini X, Wenguang C, Weimin Z. CprFS: A User-level File System
  to Support Consistent File States for Checkpoint and Restart. 22nd ACM
  International Conference on Supercomputing (ICS'08), 114-223, June 7-12, 2008.

\bibitem{tex} Knuth D~E. The {\TeX} Book. 15th ed. Reading, MA: Addison-Wesley
  Publishing Company, 1989

\bibitem{xue09} Ruini X, Xuezheng L, Ming W, Zhenyu G, Wenguang C, Weimin Z,
  Zheng Z, Geoffery M~V. MPIWiz: Subgroup Reproducible Replay of MPI
  Applications. 14th ACM SIGPLAN Symposium on Principles and Practice of
  Parallel Programming (PPoPP'09), 251-260, February 14-18, 2009.

\bibitem{Chafik94} {Chafik El Idrissi} M, Roney A, Frigon C, et~al. Measurements
  of total kinetic-energy released to the {$N=2$} dissociation limit of {H}$_2$
  --- evidence of the dissociation of very high vibrational {R}ydberg states of
  {H}$_2$ by doubly-excited states. Chemical Physics Letters, 1994,
  224(10):260--266

\bibitem{MellingerR96} Mellinger A, Vidal C~R, Jungen C.  Laser reduced
  fluorescence study of the carbon-monoxide nd triplet {R}ydberg
  series-experimental results and multichannel quantum-defect analysis.
  J. Chem. Phys., 1996, 104(5):8913--8921

\bibitem{NPB2} Woo A, Bailey D, Yarrow M, et~al. The {NAS} Parallel Benchmarks
  2.0. Technical report, The Pennsylvania State University CiteSeer Archives,
  December~05, 1995. \url{http://www.nasa.org/}
\end{thebibliography}
\end{document}
